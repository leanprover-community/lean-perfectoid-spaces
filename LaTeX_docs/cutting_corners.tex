\documentclass{amsart}
\usepackage{amsmath}
\usepackage{amsthm}
%\usepackage{a4wide}
%\usepackage{enumerate}

\newcommand{\R}{\mathbb{R}}
\newcommand{\Z}{\mathbb{Z}}
\DeclareMathOperator{\Spa}{Spa}
\DeclareMathOperator{\Spv}{Spv}
\DeclareMathOperator{\pre}{pre}

\theoremstyle{plain}
\newtheorem{theorem}{Theorem}
\newtheorem{lemma}[theorem]{Lemma}
\newtheorem{corollary}[theorem]{Corollary}
\newtheorem{proposition}[theorem]{Proposition}
\theoremstyle{remark}
\newtheorem{remark}[theorem]{Remark}
\newtheorem*{remarkn}{Remark}

\begin{document}

\title{Projective limits.}
\maketitle

\section{The set-up}

Let $X$ be a type and let $R$ be a preorder on $X$ (that is, a binary relation which is reflexive and transitive). Corresponding to $(X,R)$ is a ``small thin category'' (``small'' means the objects form a set, ``thin'' apparently means that there's at most one morphism between any two objects). We have to make a choice here: let's be covariant (it doesn't matter). So if $x\ R\ y$ then there's a morphism from $x$ to $y$.

If now $i$ is a (covariant) functor from $(X,R)$ to some category $\mathcal{C}$ then for all $x\in X$, $A_x:=i(x)$ is an object of $\mathcal{C}$, and for all $x\ R\ y$ we have a morphism $f_{xy}:=i(x\ R\ y)$ from $A_x$ to $A_y$. If $\mathcal{C}$ has all limits, then one can take the (projective) limit of $i$; in the category of commutative rings or types this would be the subring of the product ring $\prod_{x\in X}A_x$ consisting of elements $(a_x)_{x\in X}$ such that $f_{xy}(a_x)=a_y$.

{\it TODO.} If $\mathcal{C}$ is the category of topological rings, is the projective limit still the same thing, equipped with subspace topology of the product topology? Surely this is true but I never checked. I will remark that a sheaf of topological rings is not just a presheaf of topological rings such that the underlying presheaf of rings is a sheaf; I believe one needs the ring isomorphism between $\mathcal{F}(U)$ and the relevant subspace of $\prod_i\mathcal{F}(U_i)$ to be a homeomorphism, not just continuous.

Now say $S$ is a sub-preorder of $R$, that is, $S$ is another preorder on $X$ and $x\ S\ y\implies x\ R\ y$. This can certainly happen -- for example every equivalence relation is a preorder, so just imagine splitting up $R$'s equivalence classes into smaller ones. As another example one can imagine the usual total order on $\{1,2,3\}$ and then a subrelation where $1\leq 3$ and $2\leq 3$ but $1$ is no longer at most 2.

There's a natural functor $(X,S)\to(X,R)$. So the functor $i:(X,R)\to\mathcal{C}$ which sends $x$ to $A_x$ and sends the morphism $x\ R\ y$ to $f_{xy}:A_x\to A_y$, induces a functor $j:(X,S)\to\mathcal{C}$ which still sends $x$ to $A_x$ and $x\ S\ y$ to $f_{xy}:A_x\to A_y$. If $\mathcal{C}$ has all small limits one could ask what the relationship between the limits $A_R$ of $i$ and $A_S$ of $j$ is. First a thought experiment: in the category of types one can see that there is a natural map from $A_R$ to $A_S$; the $j$-limit is a (possibly) bigger subset of the same product. But of course general nonsense gives you a morphism from $A_R$ to $A_S$ in general; as part of the package of the universal object $A_R$ we know that for all $x\in X$ there's a map from $A_R$ to $A_x$, and these maps commute with all the $f_{xy}$ for $x\ R\ y$ and hence for all the $f_{xy}$ with $x\ S\ y$; now by the universal property of $A_S$ we get a map from $A_R$ to $A_S$.

In this note I'll explain a criterion for this map to be an isomorphism in $\mathcal{C}$.

\section{The criterion.}

Say the sub-preorder $S$ has the following property: For all $x,y\in X$, if $x\ R\ y$ then there exists $z\in X$ such that $x\ S\ z$, $y\ S\ z$ and $z\ R\ y$. What does this buy us? Well, $y\ S\ z$ implies $y\ R\ z$, and then $z\ R\ y$ implies that $y$ and $z$ are $R$-isomorphic and hence $f_{yz}$ is an isomorphism in $\mathcal{C}$. In particular even though it might not be true that $x\ S\ y$, it is true that $x\ S\ z$ and that $y$ and $z$ are $R$-isomorphic. In particular the morphism $f_{xy}:A_x\to A_y$ coming from $R$ is isomorphic in $\mathcal{C}$ to the $S$-morphism $f_{xz}:A_x\to A_z$.

I claim that if $S$ has this property, then the $i$-limit $A_R$ and the $j$-limit $A_S$ are isomorphic. The fact that $S\subseteq R$ gives us a map from the $i$-limit to the $j$-limit as we saw above; what we need is a map from the $j$-limit to the $i$-limit. We want to get this map by using the universal property of the $i$-limit; to do this we need to check that the $j$-limit $A_S$ equipped with its maps $\pi_x:A_S\to A_x$ satisfies $f_{xy}\circ\pi_x=\pi_y$ for all $x,y$ with $x\ R\ y$. What we know (as part of the package of the universal object $A_S$) is that this is true if $x\ S\ y$. But now say $x\ R\ y$. By our hypothesis on~$S$ we can choose $z$ with $x\ S\ z$, $y\ S\ z$ and $z\ R\ y$; hence $f_{yz}$ is an isomorphism and to prove $f_{xy}\circ\pi_x=\pi_y$ it to prove this after applying $f_{yz}$.

It thus suffices to prove $f_{yz}f_{xy}\pi_x=f_{yz}\pi_y$; but the left hand size is $f_{xz}\pi_x$ because $i$ is a functor, and hence both sides are $f_z$ because $x\ S\ z$ and $y\ S\ z$. Hence both sides are equal and by the universal property of $A_R$ we get a map $A_S\to A_R$. The composite maps in both directions commute with all triangles formed by the maps $A_R\to A_x$ and $A_S\to A_x$ so must be the identity map by uniqueness.

\end{document}
