\documentclass{amsart}
\usepackage{amsmath}
\usepackage{amsthm}
%\usepackage{a4wide}
%\usepackage{enumerate}

\newcommand{\Q}{\mathbb{Q}}
\newcommand{\R}{\mathbb{R}}
\newcommand{\Z}{\mathbb{Z}}
\DeclareMathOperator{\Spa}{Spa}

\theoremstyle{plain}
\newtheorem{theorem}{Theorem}
\newtheorem{lemma}[theorem]{Lemma}
\newtheorem{corollary}[theorem]{Corollary}
\newtheorem{proposition}[theorem]{Proposition}
\theoremstyle{remark}
\newtheorem{remark}[theorem]{Remark}
\newtheorem*{remarkn}{Remark}

\title{Overview of definition of an adic space.}
\begin{document}

\maketitle

This document was written on 17th Sept 2018.

\section{Huber rings.}

A \emph{Huber ring} is a topological ring $A$ satisfying the following additional axiom: there exists an open subring $A_0$ containing a finitely-generated ideal $J$ such that the induced topology on $A_0$ is the $J$-adic one. Note that $A_0$ is not part of the data.

Note: this terminology is due to Scholze. The older terminology for this notion is an ``$f$-adic ring''.

A basic example is the $p$-adic numbers $\Q_p$ with its usual topology, with subring $\Z_p$ having topology generated by the ideal $(p)$.

Equivalent definitions of a Huber ring, and basic theorems about these rings, can be found in Proposition/Definition 6.1 of \cite{Wedhorn} and the lemmas following it.

The notion of a Huber ring has been formalised in the perfectoid project, in the file {\tt Huber\_ring.lean}.

\section{Huber pairs.}

A \emph{Huber pair} is a pair $(A,A^+)$ consisting of a Huber ring $A$ and an open integrally-closed subring of power-bounded elements. Again this is Scholze's terminology -- these used to be called ``affinoid rings''. 

Details of what these words mean, and some basic properties of Huber pairs, can be found in Definition~7.14 of~\cite{Wedhorn}.

An example of a Huber pair is the pair $(\Q_p,\Z_p)$. Another example is $(\Q_p[T],\Z_p[T])$ with $\Z_p[T]$ given the $p$-adic topology (that is, a polynomial is small if and only if all its coefficents are $p$-adically small -- we don't mind if the degree is big or if the constant coefficient is non-zero).

The notion of a Huber pair has been formalised in the perfectoid project, in the file {\tt Huber\_pair.lean}.

\section{The adic spectrum of a Huber pair.}

The \emph{adic spectrum} $\Spa(A)$ of a Huber pair $(A,A^+)$ (note abuse of notation; $A^+$ is often not mentioned even though it is not uniquely determined by $A$) is quite an elaborate object.

As a set (or a type, if you prefer), it is just an equivalence class of valuations on $A$ with some properties. Valuations and the equivalence relation on them are defined in \cite{Wedhorn} chapter 1, and these notions have been formalised in Lean and the type {\tt Spa A} for $A$ a Huber pair is defined in {\tt Spa.lean}.

This set comes with a bunch of extra structure, not all of which is currently in Lean.

1) It has the structure of a topological space, and amongst the open subsets are a special kind of open subsets called \emph{rational subsets}.

2) The topological space has a presheaf of complete topological rings on it. The presheaf is first defined on the rational subsets. To make the definition on the rational subsets one has to have localisations of rings (which mathlib has), and completions of topological rings, which it does not yet quite have (but we are nearly there, thanks to Patrick Massot's valiant efforts). The definition of this presheaf in full is in section 8.1 of \cite{Wedhorn}.

3) Furthermore, the stalks of (the underlying presheaf of rings associated to)this presheaf of topological rings, which can be shown to be local rings, are then each equipped with an equivalence class of valuations, whose support is the maximal ideal. These valuations should not be difficult to define, but we have not done it yet.

{\bf Current status}: We need to be able to complete a topological ring and prove that the completion has the universal property stated in 5.32 of \cite{Wedhorn}. This is blocking (2). After that I am hoping (3) should not be too hard, as we have a bunch of stuff on valuations, but as far as I know nobody has thought about this.

\section{Adic spaces}

An adic space is a topological space with a sheaf of topological rings, such that the stalks (in the underlying sheaf of rings) are local, and each of these is equipped with a valuation whose support is the maximal ideal, which has an open cover by subsets each isomorphic to $Spa(A)$ for some Huber pair $A$.

Key definitions in \cite{Wedhorn}. $V^{pre}$ is on p76, $\mathcal{V}$ on p80, adic space is definition 8.21.

\section{Perfectoid spaces}

A perfectoid space is just an adic space with some extra property. All this is already in Lean.

\end{document}
