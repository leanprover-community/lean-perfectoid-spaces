\documentclass{amsart}
\usepackage{amsmath}
\usepackage{amsthm}
%\usepackage{a4wide}
%\usepackage{enumerate}

\newcommand{\R}{\mathbb{R}}
\newcommand{\Z}{\mathbb{Z}}
\DeclareMathOperator{\Spa}{Spa}
\DeclareMathOperator{\Spv}{Spv}
\DeclareMathOperator{\pre}{pre}

\theoremstyle{plain}
\newtheorem{theorem}{Theorem}
\newtheorem{lemma}[theorem]{Lemma}
\newtheorem{corollary}[theorem]{Corollary}
\newtheorem{proposition}[theorem]{Proposition}
\theoremstyle{remark}
\newtheorem{remark}[theorem]{Remark}
\newtheorem*{remarkn}{Remark}

\begin{document}

\title{Adic spaces.}

\section{$\mathcal{V}^{\pre}$}

The definition of an adic space is on Wedhorn p80. The category $\mathcal{V}$ is defined on p80 and the category $\mathcal{V}^{\pre}$ is defined on p76.

The category $\mathcal{V}^{\pre}$ is defined as follows. An object is a topological space, plus a presheaf of complete topological rings, such that the stalks (direct limit in the category of rings, with no topology) are local, and also each stalk is equipped with an equivalence class of valuations such that the support of the valuation is the maximal ideal. We know that supports are constant across equivalence classes so there is no obstruction to making this definition in Lean.

Oh wait -- do we have direct limits of rings? I think there was some hold-up getting them into mathlib.

TODO : direct limits of commutative rings (just for a directed set)

A morphism in $\mathcal{V}^{\pre}$ is a continuous map $f:X\to Y$ plus a morphism $f^\flat:\mathcal{O}_Y\to f_*\mathcal{O}_X$ of presheaves, such that the induced map $f^\flat_x:\mathcal{O}_{Y,f(x)}\to\mathcal{O}_{X,x}$ has the property that the valuation on the stalk at $x$ pulls back (up to equivalence) to the valuation on $f(x)$.

TODO : universal property of direct limits

TODO : pushforward of presheaf (NB : this is the easy one).

TODO : composition of morphisms is a morphism, identity morphism is a morphism, definition of isomorphism.

If $X$ + structure is an object in $\mathcal{V}^{\pre}$ and $U$ is an open subset of $X$ (or more generally I think we should let $U\to X$ be an open immersion) then the pullback of everything to $U$ should be an object in $\mathcal{V}^{\pre}$ too.

TODO : define this construction. tedious check that stalks don't change etc.

The category $\mathcal{V}$ is just the full subcategory of $\mathcal{V}^{\pre}$ consisting of objects for which the presheaf is a sheaf.

TODO : definition of sheaf.

\section{$\Spa(A)$}

Last time I checked, we had a sorry-free definition of a Huber Pair $A=(R,R^+)$. If $A$ is a Huber pair, then it would be nice to construct an object of $\mathcal{V}^{\pre}$ associated to this data. However, this is annoyingly difficult because there is a bunch of API for Cont (theorem 3.3 of Huber's paper ``continuous valuations'') which I propose skipping. Here are the details.

We have a topological space structure on $\Spv(R)$, the set of all valuations on $R$, and $\Spa(A)$ is the subspace of continuous valuations which are bounded by 1 on $R^+$. We already have this. The topology is generated by basic opens (which we have), and rational opens are open (we have this).

TODO : rational opens are a basis [NB : do we use this?]

Now let's talk about the presheaf. First we need some ring constructions. If $s\in R$ and $T\subseteq R$ is a finite subset generating an open ideal in $R$, then there is a rational open subset of $\Spa(R)$ which Wedhorn calls $R(T/s)$, but following general alg geom conventions let me call it $D(T/s)$ instead. The idea is that the value of the presheaf on $D(T/s)$ is a ring called $R\langle T/s\rangle$ and then we extend the presheaf to a general open by taking projective limits.

TODO : Johan is in the process of defining $R(T/s)$ and proving that it is a Huber ring. We also need that it's an $R$-algebra.

TODO : Prove the unversal property of $R(T/s)$ as in Wedhorn 5.51

TODO : Define $R\langle T/s\rangle$ to be the completion of this ring. I believe Patrick has pushed enough of ring closures to be able to do this. We also need that it's an $R$-algebra, which just boils down to having a map from a ring to its completion.

TODO : Prove the universal property of $R\langle T/s\rangle$ as descrived on Wedhorn p74 just before Lemma 8.1 -- this should just follow from the universal property of completion and the universal property of $R(T/s)$.

Johan has defined {\tt rational\_open (s : A) (T : set A)} to be $D(T/s)$.

Note that $D(T/s)=D(T\cup\{s\}/s)$ so WLOG $s\in T$, and if $s_1\in T_1$ and $s_2\in T_2$ then Johan has proved that $D(T_1/s_1)\cap D(T_2/s_2)=D(T_1T_2/s_1s_2)$. There is also a map $A\langle T_1/s_1\rangle \to A\langle T_1T_2/s_1s_2\rangle$, which follows immediately from the universal property for completion.

Note: it is true that if $D(T_1/s_1)\subseteq D(T_2/s_2)$ then there's a natural map $A\langle T_2/s_2\rangle \to A\langle T_1/s_1\rangle$. However, to deduce this from the universal property is quite a substantial amount of work, as far as I can see, because it relies on Wedhorn 7.52 which relies on Wedhorn 7.18 which relies on Huber's Continuous valuations paper and also a bunch of stuff about specialising and generalising valuations.

However, we can still define the presheaf! We would like to define its value on an open $U\subseteq\Spa(A)$ to be the projective limit of the $R\langle T/s\rangle$ as $(T,s)$ range over the pairs with the usual property ($T$ is finite and generates an open ideal) with $D(T/s)\subseteq U$. The problem is with the transition maps; if $a_1\in D(T_1/s_1)$ and $a_2\in D(T_2/s_2)$ and if $D(T_2/s_2)\subseteq D(T_1/s_1)$ then we want to say that $a_1$ gets mapped to $a_2$ when we apply the map $R\langle T_1/s_1\rangle\to R\langle T_2/s_2\rangle$ which we can't construct. However we can use the following trick. WLOG $s_i\in T_i$. Then we can construct the maps $R\langle T_1/s_1\rangle\to R\langle T_1T_2/s_1s_2\rangle$ and $R\langle T_2/s_2\rangle\to R\langle T_1T_2/s_1s_2\rangle$ from the universal property. This latter map is known to be an isomorphism, even though we can't prove it. However it does mean that if we demand that $a_1$ and $a_2$ have the same image in $R\langle T_1T_2/s_1s_2\rangle$ for all $(T_i,s_i)$ then the corresponding subset of the product will be provably (in maths) the projective limit and hence the correct object.

My proposal for the definition of the value of the presheaf on an open $U$ is hence the subset of $\prod_{(T,s)\,:\,D(T/s)\subseteq U}R\langle T/s\rangle$ consisting of elements $a_{(T,s)}$ such that $a_{(T_1,s_1)}$ and $a_{(T_2,s_2)}$ agree in $R\langle T_1T_2/s_1s_2\rangle$. It's a theorem of maths that this is (canonically) isomorphic to the value of the correct presheaf on $U$.

TODO : Check that this closed (because $R\langle T/s\rangle$ is Hausdorff) subring of an arbitrary product of complete topological rings is complete (ask Patrick!). Is this something to do with reflective subcategories?

Note that this is trivially checked to be a presheaf.

TODO : check it's a presheaf.

Now we need to put a valuation on the stalks. The way this works is that if $v$ is a continuous valuation on $R$ then there's a map $R\to K_v$ and a valuation on $K_v$ extending $v$. Now say $(T,s)$ are as usual with $v\in D(T/s)$; then $v(s)$ is non-zero so $s\in K_v^\times$, and for $t\in T$ we have $v(t)\leq v(s)$, so $t/s\in K_v$ has valuation at most~1. By the universal property of $R(T/s)$, we get a map $R(T/s)\to K_v$ and hence a valuation on $R(T/s)$. We will also need to prove a uniqueness statement to make sure that the valuations glue. Note that the valuation group doesn't go up. This still needs some thought.

TODO : There's a lemma here, which we need to prove. A continuous valuation on $R$ extends uniquely to a continuous valuation on its completion! I haven't thought about this and it's essential. Huber says it's ``clear''.{\bf Note that this needs to be checked -- the proof in Wedhorn uses a lot of machinery which we don't have and proves something much stronger.}

TODO : valuation on the factors gives a valuation on the direct limit.

Now to prove that this is an object in $\mathcal{V}^{\pre}$ we would need to prove that stalks are local and that the support of the valuation is the maximal ideal. This is true in maths but I propose skipping the Lean proofs because they I think will involve Huber 3.3. However we do not need these proofs -- we are just making a definition.

\section{Adic spaces}

An adic space is an object in $\mathcal{V}$ which has an open cover by subsets each isomorphic to $\Spa(A)$ for some Huber pair $A$. The isomorphism takes place in $V^{\pre}$, but we do not need these proofs which we just skipped : an isomorphism of objects in $\mathcal{V}^{\pre}$ is just an isomorphism of topological spaces and an isomorphism of presheaves which induces an isomorphism of stalks such that the valuation on one is equivalent to the pullback of the valuation on the other.

I guess formally one could say this as follows. The category $\mathcal{V}^{\pre}$  is a full subcategory of a bigger category where we drop the things that we cannot prove -- we drop the demand that the stalks are local and we drop the demand that the valuations on the stalks have support equal to the maximal ideal. Let $\mathcal{C}$ denote this bigger category. We still demand that the morphisms are compatible with the valuations on the stalks. We can prove that $\Spa(A)$ is an object of $\mathcal{C}$. But because isomorphisms in $\mathcal{V}^{\pre}$ have the same definition as isomorphisms in $\mathcal{C}$ we can still define an adic space to be an object of $\mathcal{V}$ which has an open cover such that the induced objects of $\mathcal{C}$ are isomorphic in $\mathcal{C}$ to things of the form $\Spa(A)$.



\end{document}
